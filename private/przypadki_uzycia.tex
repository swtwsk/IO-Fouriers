\documentclass{article}

% SETTINGS %
\usepackage{polski}
\usepackage[utf8]{inputenc}
\usepackage{titlesec}
\usepackage[bookmarks=true,hidelinks]{hyperref}

\newcommand{\sectionbreak}{\clearpage}
\newcommand{\documentdate}{23 marca 2018}
\newcommand{\documentversion}{ver. 1.1}

\title{Fourier's Phone - przypadki użycia}
\author{Frederic Grabowski \and Bartłomiej Karasek \and Wojciech Przybyszewski 
        \and Andrzej Swatowski}
\date{\documentdate \\ \documentversion}

%------------------
% DOCUMENT %
\begin{document}

\maketitle
\newpage

\tableofcontents
\newpage

\section{Lista przypadków użycia}
    \begin{enumerate}
        \item Przesyłanie niewielkich plików (P2P)
        \item Demonstrowanie komunikacji w sieciach
        \item Przesyłanie wiadomości tekstowych (party trick/gag)
        \item Wykonywanie performance'ów artystycznych (modern art)
        \item Antropologiczny komentarz społeczeństwa XXI wieku
        \item Łodzie podwodne i wieloryby
    \end{enumerate}

\newpage
\section{Przesyłanie niewielkich plików (P2P)}
\subsection{Krótki opis}
Ten przypadek opisuje w jaki sposób użytkownicy aplikacji mogą bezpośrednio wymieniać między sobą niewielkie pliki.
\subsection{Aktorzy}
\subsubsection{Nadawca}
\subsubsection{Odbiorcy (co najmniej jeden)}
\subsection{Wymagania}
Aktorzy znajdują się we względnie cichym miejscu z ośrodkiem umożliwiającym rozchodzenie się fal dźwiękowych (na przykład \bf{szafa}).

\subsection{Scenariusz użycia}
\begin{enumerate}
	\normalfont \item \bf Nadawca \normalfont wybiera plik do wysłania i informuje \bf odbiorców \normalfont o intencji wysłania pliku.
	\item \bf Odbiory \normalfont włączają aplikację w trybie odbierania przygotowując się mentalnie i sprzętowo do transmisji.
	\item \bf Nadawca \normalfont po upewnieniu się o gotowości \bf odbiorców \normalfont włącza aplikację w trybie nadawania i następnie rozpoczyna transmisję.
	\item \bf Nadawca \normalfont i \bf odbiorcy \normalfont zachowują ciszę i spokój do czasu zakończenia transmisji o którym zostaną poinformowani przez aplikację.
	\item \bf Odbiorcy \normalfont zapisują odebrany plik u siebie na urządzeniach.
\end{enumerate}

\subsection{Potencjalne problemy}
\subsubsection{Niekompletny odbiór}
\normalfont
Jeżeli jeden z odbiorców rozpocznie odbieranie transmisji za późno lub nadawca rozpocznie nadawanie za wcześnie odebrany plik może być uszkodzony. Dla bezpieczeństwa zalecane jest powtórzenie transmisji oraz niezapisywanie odebranego pliku.
\subsubsection{Zakłócenie odbioru}
\normalfont
Jeżeli nadawca i odbiorca dojdą do wniosku że z powodu zanieczyszczenia hałasem transmisja mogła się nie udać zaleca się postępowanie jak w punkcie \bf Niekompletny odbiór \normalfont


\newpage
\section{Wykonywanie performance'ów artystycznych (modern art)}
\subsection{Krótki opis}
Ten przypadek opisuje w jaki sposób artyści za pomocą aplikacji mogą tworzyć sztukę współczesną lub wykonywać performance'y.
\subsection{Aktorzy}
\subsubsection{Sztukmistrz}
\subsubsection{Widzowie (opcjonalni)}
\subsection{Wymagania}
Aktorzy znajdują się we względnie cichym miejscu z ośrodkiem umożliwiającym rozchodzenie się fal dźwiękowych (na przykład \bf{szafa}).

\subsection{Scenariusz użycia}
\normalfont
Oczywiście jest to jedynie przykład; zachęcamy do poszukiwania własnych środków \bf wyrazu artystycznego\normalfont{.}
\begin{enumerate}
	\normalfont \item \bf Sztukmistrz \normalfont wchodzi do \bf szafy \normalfont
	\item \bf Sztukmistrz \normalfont uruchamia pierwsze urządzenie w trybie nadawania
	\item \bf Sztukmistrz \normalfont uruchamia drugie urządzenie w trybie nadawania wybierając \it Pana Tadeusza \normalfont
	\item Po przepuszczeniu wynikowej wiadomości przez autokoretkę, improwizuje przedstawienie oparte o powstały tekst
	\item \bf Widzowie \normalfont są \bf zachwyceni \normalfont
\end{enumerate}

\subsection{Potencjalne problemy}
\normalfont
Nie podejmujemy się komentarza tak subiektywnego doświadczenia. Każdą słabość można przekuć w siłę.


\end{document}