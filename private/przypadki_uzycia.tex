\documentclass{article}

% SETTINGS %
\usepackage{polski}
\usepackage[utf8]{inputenc}
\usepackage{titlesec}
\usepackage[bookmarks=true,hidelinks]{hyperref}

\newcommand{\sectionbreak}{\clearpage}
\newcommand{\documentdate}{30 marca 2018}
\newcommand{\documentversion}{ver. 1.2}

\title{Fourier's Phone - przypadki użycia}
\author{Frederic Grabowski \and Bartłomiej Karasek \and Wojciech Przybyszewski 
        \and Andrzej Swatowski}
\date{\documentdate \\ \documentversion}

%------------------
% DOCUMENT %
\begin{document}

\maketitle
\newpage

\tableofcontents
\newpage

\section{Lista przypadków użycia}
    \begin{enumerate}
        \item Przesyłanie plików (P2P)
        \item Demonstrowanie komunikacji w sieciach
        \item Przesyłanie wiadomości tekstowych (party trick/gag)
        \item Wykonywanie performance'ów artystycznych (modern art)
        \item Komunikacja podwodna (z łodziami podwodnymi, wielorybami itp.)
    \end{enumerate}

\newpage
\section{Przesyłanie plików (P2P)}
\subsection{Krótki opis}
Ten przypadek opisuje w jaki sposób użytkownicy aplikacji mogą bezpośrednio wymieniać między sobą niewielkie pliki.
\subsection{Aktorzy}
\subsubsection{Nadawca}
\subsubsection{Odbiorcy (co najmniej jeden)}
\subsection{Wymagania}
Aktorzy znajdują się we względnie cichym miejscu z ośrodkiem umożliwiającym rozchodzenie się fal dźwiękowych (na przykład \bf{szafa}).

\subsection{Scenariusz użycia}
\begin{enumerate}
	\normalfont \item \bf Nadawca \normalfont wybiera plik do wysłania i informuje \bf odbiorców \normalfont o intencji wysłania pliku.
	\item \bf Odbiory \normalfont włączają aplikację w trybie odbierania przygotowując się mentalnie i sprzętowo do transmisji.
	\item \bf Nadawca \normalfont po upewnieniu się o gotowości \bf odbiorców \normalfont włącza aplikację w trybie nadawania i klika przycisk rozpoczęcia transmisji.
	\item Aplikacja rozpoczyna nadawanie pliku przy użyciu dźwięków słyszalnych.
	\item \bf Nadawca \normalfont i \bf odbiorcy \normalfont zachowują ciszę i spokój do czasu zakończenia transmisji, o czym zostaną poinformowani przez aplikację.
	\item \bf Odbiorcy \normalfont zostają przez aplikację poinformowani o zakończeniu odbierania i korzystając z niej zapisują w wybranej lokalizacji plik.
\end{enumerate}

\subsection{Potencjalne problemy}
\subsubsection{Niekompletny odbiór}
\normalfont
Jeżeli jeden z odbiorców rozpocznie odbieranie transmisji za późno lub nadawca rozpocznie nadawanie za wcześnie mechanizmy zabezpieczające (oddzielne dźwięki startu i końca nadawania, sumy kontrolne) powinny doprowadzić do nieodebrania żadnego pliku. W takim przypadku zaleca się retransmisję.
\subsubsection{Zakłócenie odbioru}
\normalfont
Jeżeli nadawca i odbiorca dojdą do wniosku że z powodu zanieczyszczenia hałasem mimo, iż plik doszedł, to może być uszkodzony, zaleca się niezapisywanie pliku i retransmisję.


\newpage
\section{Wykonywanie performance'ów artystycznych (modern art)}
\subsection{Krótki opis}
Ten przypadek opisuje w jaki sposób artyści za pomocą aplikacji mogą tworzyć sztukę współczesną lub wykonywać performance'y.
\subsection{Aktorzy}
\subsubsection{Sztukmistrz}
\subsubsection{Widzowie (opcjonalni)}
\subsection{Wymagania}
Aktorzy znajdują się we względnie cichym miejscu z ośrodkiem umożliwiającym rozchodzenie się fal dźwiękowych (na przykład \bf{szafa}).

\subsection{Scenariusz użycia}
\normalfont
Oczywiście jest to jedynie przykład; zachęcamy do poszukiwania własnych środków \bf wyrazu artystycznego\normalfont{.}
\begin{enumerate}
	\normalfont \item \bf Sztukmistrz \normalfont wchodzi do \bf szafy \normalfont.
	\item \bf Sztukmistrz \normalfont uruchamia pierwsze urządzenie w trybie odbierania.
	\item \bf Sztukmistrz \normalfont uruchamia drugie urządzenie w trybie nadawania wybierając \it Pana Tadeusza \normalfont.
	\item \bf Sztukmistrz \normalfont zaczyna śpiewać \it Odę do Radości \normalfont zakłócając tym samym transmisję.
	\item Po przepuszczeniu wynikowej wiadomości przez autokorektę, improwizuje przedstawienie oparte o powstały tekst.
	\item \bf Widzowie \normalfont są \bf zachwyceni\normalfont.
\end{enumerate}

\subsection{Potencjalne problemy}
\normalfont
Nie podejmujemy się komentarza tak subiektywnego doświadczenia. Każdą słabość można przekuć w siłę.


\end{document}
