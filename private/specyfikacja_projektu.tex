\documentclass{article}

% SETTINGS %
\usepackage{polski}
\usepackage[utf8]{inputenc}
\usepackage{titlesec}
\usepackage[bookmarks=true,hidelinks]{hyperref}
\usepackage{xcolor}
\usepackage{color,soul}

\newcommand{\sectionbreak}{\clearpage}
\newcommand{\documentdate}{30 marca 2018}
\newcommand{\documentversion}{ver. 1.2}
\newcommand{\seriaA}[1]{\color{pink}{#1}\color{black}}
\newcommand{\seriaB}[1]{\color{blue}{#1}\color{black}}

\title{Fourier's Phone - szczegółowa specyfikacja}
\author{Frederic Grabowski \and Bartłomiej Karasek \and Wojciech Przybyszewski 
        \and Andrzej Swatowski}
\date{\documentdate \\ \documentversion}

%------------------
% DOCUMENT %
\begin{document}

\maketitle
\newpage

\tableofcontents
\newpage

\section{Szczegółowe założenia działania projektu}
    \begin{enumerate}
        \item \seriaA{Aplikacja umożliwia przesyłanie danych między telefonami przy użyciu:}
            \begin{enumerate}
                \item \seriaA{dźwięku.}
                \item światła.
            \end{enumerate}
        \item Aplikacja umożliwia przesyłanie określonych typów danych:
            \begin{enumerate}
                \item \seriaB{przesyłanie plików.}
                \item \seriaA{przesyłanie tekstu.}
            \end{enumerate}
        \item Aplikacja może działać także w module komunikatora tekstowego.
        \item \seriaA{Telefon może działać w trybie urządzenie nadającego lub urządzenia
              odbierającego.}
        \item \seriaB{Aplikacji użytkownik wybiera za pomocą menadżera plików plik do
              przesłania:
            \begin{enumerate}
            \item {Menadżer plików ma dostęp tylko do plików nieukrywanych w~żaden 
                  sposób przez system telefonu.}
            \end{enumerate}}
        \item \seriaA{Po wybraniu pliku telefon rozpoczyna nadawanie pliku i~staje się 
              urządzeniem nadającym.}
            \begin{enumerate}
                \item \seriaA{Aplikacja nie wykrywa, czy w zasięgu nadawania urządzenia
                      nadającego znajduje się jakieś urządzenie odbierające.}
                \item \seriaA{Aplikacja nie podaje również żadnych informacji na temat
                      sukcesu nadawania.}
                \item \seriaA{W przypadku przerwania nadawania (przez użytkownika za 
                      pomocą przycisku lub z~powodów niezależnych od użytkownika) 
                      nie jest określone, czy urządzenie odbierające odbierze 
                      choć część pliku.}
                \item \seriaA{Podobnie w przypadku zakłóceń zewnętrznych (szumów, krzyków,
                      itp.) nie ma gwarancji, że nawet w~przypadku powodzenia 
                      nadawania urządzenie odbierające poprawnie odbierze plik.}
            \end{enumerate}
        \item \seriaA{Po zakończeniu nadawania urządzenie nadające poinformuje 
              o~zakończeniu nadawania.}
        \item \seriaB{W~aplikacji można wybrać opcję odbierania pliku przez telefon, 
              który staje się wtedy urządzeniem odbierającym.}
        \item \seriaA{Po wybraniu opcji odbierania urządzenie odbierające rozpoczyna 
              nasłuchiwanie, czy odbywa się jakieś nadawanie.}
        \item \seriaA{W~przypadku nie znalezienia urządzenia nadającego w~ciągu 
              ustawionego czasu (domyślnie 10 sekund) użytkownik zostaje o~tym 
              fakcie poinformowany, a~odbieranie zostaje przerwane.}
        \item \seriaA{W~przypadku znalezienia choć jednego urządzenia odbierającego, 
              rozpoczyna się odbieranie.}
            \begin{enumerate}
                \item \seriaA{W~sytuacjach opisanych w~punktach 6.c. i~6.d., a~także 
                      w~sytuacji, gdy jest więcej niż jedno urządzenie nadające 
                      w~zasięgu, nie ma gwarancji, czy urządzenie odbierze plik 
                      poprawnie.}
                \item \seriaA{Urządzenie odbierające może w~przypadku takich problemów 
                      poinformować o~błędzie odbierania i~zakończyć działanie 
                      w~trybie odbierania.}
                \item \seriaA{Urządzenie odbierające może również odebrać plik niepoprawny.}
            \end{enumerate}
        \item \seriaB{Po zakończeniu odbierania urządzenie prosi o~podanie nazwy pliku, 
              pod którą zapisze odebrany plik w~katalogu utworzonym przez aplikację. \\
              W~przypadku niepoprawnej (niezgodnej z~przyjętym standardem nazywania 
              plików w~systemach Android) nazwy użytkownik zostanie poproszony 
              o~inną nazwę.}
        \item Moduł komunikatora korzysta ze specjalnie zdefiniowanego protokołu, 
              który pozwala nawiązać połączenie między urządzeniami i~umożliwiające
              wzajemną komunikację. Protokół ten umożliwia komunikowanie się 
              między wieloma urządzeniami, bez konieczności ręcznej zmiany trybu 
              przez użytkownika aplikacji.
    \end{enumerate}
    
%\section{Szczegółowa specyfikacja protokołu}
%    \begin{enumerate}
%        \item Protokół jest zbiorem ścisłych reguł i~kroków postępowania, które są
%              automatycznie wykonywane przez urządzenia komunikacyjne w~celu nawiązania
%              łączności i~wymiany danych.
%        \item Aplikacja musi wykonać wszystkie opisane niżej kroki, by nawiązać
%              łączność z~inną aplikacją.
%        \item Protokół wymaga od użytkownika wybrania początkowego trybu - odbierania lub
%              nadawania. W~architekturze \textit{klient-serwer} jest to z~góry ustalone,
%              w~przypadku korzystania z~architektury \textit{peer-to-peer} ustalają
%              to użytkownicy aplikacji przed nawiązaniem połączenia.
%        \item Za pomocą protokołu komunikować mogą się jedynie dwa urządzenia.
%        \item Protokół składa się z czterech wyszczególnionych części:
%            \begin{enumerate}
%                \item procedury powitalnej (tak zwany \textit{handshake}):
%                    \begin{enumerate}
%                        \item Urządzenie w~trybie nadawania wysyła do urządzenia 
%                              odbierającego określoną sekwencję sygnałów dźwiękowych, 
%                              oznaczającą \textbf{HAIL} oraz \textbf{dwa sygnały 
%                              wyznaczające szerokość pasma}, jakie zamierza wykorzystywać do 
%                              komunikacji - najniższy możliwy dźwięk i~najwyższy 
%                              możliwy dźwięk.
%                        \item Urządzenie odbierające odbiera wyżej wymienioną
%                              sekwencję i~musi w~ciągu 30 sekund odpowiedzieć
%                              analogiczną sekwencją, oznaczającą \textbf{HAIL ACC}.
%                        \item Gdy urządzenie nadające odbierze odpowiedź urządzenia
%                              odbierającego, urządzenia zakładają, że procedura
%                              powitalna zakończyła się ustaleniem połączenia.
%                    \end{enumerate}
%                \item właściwego przekazu danych:
%                    \begin{enumerate}
%                        \item Urządzenie nadające wysyła dane w pakietach po 1024 bajty.
%                        \item Urządzenie odbierające odbiera wyżej wymienioną
%                              sekwencję i~musi w~ciągu 30 sekund odpowiedzieć
%                              analogiczną sekwencją, oznaczającą \textbf{HAIL ACC}.
%                        \item Gdy urządzenie nadające odbierze odpowiedź urządzenia
%                              odbierającego, urządzenia zakładają, że procedura
%                              powitalna zakończyła się ustaleniem połączenia.
%                    \end{enumerate}
%                \item procedury sprawdzania sum kontrolnych:
%                \item procedury pożegnania, która może zakończyć transmisję lub
%                      kontynuować ją, poprzez powrót do procedury powitalnej:
%            \end{enumerate}
%    \end{enumerate}

\section{Założenia wydajnościowe}
    \begin{enumerate}
        \item Urządzenie przesyła dane z~szybkością ok. 64 bitów na sekundę.
        \item Warunki otoczenia nie wpływają na szybkość przesyłania, a~jedynie 
              na skuteczność przesyłania.
    \end{enumerate}
        
\section{Założenia pojemnościowe}
    \begin{enumerate}
        \item Urządzenie nadające może przesyłać na raz jeden zestaw danych.
        \item Urządzenie odbierające może odbierać na raz jeden zestaw danych.
        \item W przypadku równoczesnego nadawania przez dwa urządzenia nadające 
              może nastąpić błąd odbierania, który skutkuje nieprawidłową reakcją
              urządzenia odbierającego.
        \item Równoczesne odbieranie przez dowolną ilość urządzeń odbierających 
              nie skutkuje błędem.
    \end{enumerate}
        
\section{Wymagania sprzętowe}
    \begin{enumerate}
        \item Aplikacja przeznaczona jest na telefony z systemem Android w~wersji
              5.0 lub wyższej.
        \item Telefon musi ponadto posiadać sprawnie działający głośnik i~mikrofon
              pozwalające odpowiednio nadawać i~odbierać dźwięki o~częstotliwości
              od 220 Hz do 8 kHz
    \end{enumerate}
        
\section{Wymagania środowiskowe}
    \begin{enumerate}
        \item Przesyłanie danych musi odbywać się w~cichym, niezaszumionym miejscu, 
              w~którym znajduje się ośrodek umożliwiający rozchodzenie się fal 
              dźwiękowych (np. nie w~próżni).
        \item Odległość w~jakiej mogą znaleźć się telefony, aby móc poprawnie 
              przesłać plik, zależy od sprawności działania ich głośnika i~mikrofonu
              oraz miejsca w~którym odbywa się przesyłanie danych (rozumianego 
              jako położenie w~przestrzeni i~własności fizyczne ośrodka, w~którym
              następuje wymiana danych), przy czym jest ona charakterystyczna 
              dla każdej pary telefonów, a~nie pojedynczego telefonu.
    \end{enumerate}

\end{document}
