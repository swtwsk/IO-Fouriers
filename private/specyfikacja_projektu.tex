\documentclass{article}

% SETTINGS %
\usepackage{polski}
\usepackage[utf8]{inputenc}
\usepackage{titlesec}
\usepackage[bookmarks=true,hidelinks]{hyperref}

\newcommand{\sectionbreak}{\clearpage}
\newcommand{\documentdate}{23 marca 2018}
\newcommand{\documentversion}{ver. 1.1}

\title{Fourier's Phone - szczegółowa specyfikacja}
\author{Frederic Grabowski \and Bartłomiej Karasek \and Wojciech Przybyszewski 
        \and Andrzej Swatowski}
\date{\documentdate \\ \documentversion}

%------------------
% DOCUMENT %
\begin{document}

\maketitle
\newpage

\tableofcontents
\newpage

\section{Szczegółowe założenia działania projektu}
    \begin{enumerate}
        \item Aplikacja umożliwia przesyłanie plików między telefonami przy 
              użyciu dźwięku.
        \item Telefon może działać w trybie urządzenie nadającego lub urządzenia
              odbierającego.
        \item W aplikacji użytkownik wybiera za pomocą menadżera plików plik do
              przesłania.
        \item Menadżer plików ma dostęp tylko do plików nieukrywanych w~żaden 
              sposób przez system telefonu.
        \item Po wybraniu pliku telefon rozpoczyna nadawanie pliku i~staje się 
              urządzeniem nadającym.
            \begin{enumerate}
                \item Aplikacja nie wykrywa, czy w zasięgu nadawania urządzenia
                      nadającego znajduje się jakieś urządzenie odbierające.
                \item Aplikacja nie podaje również żadnych informacji na temat
                      sukcesu nadawania.
                \item W przypadku przerwania nadawania (przez użytkownika za 
                      pomocą przycisku lub z~powodów niezależnych od użytkownika) 
                      nie jest określone, czy urządzenie odbierające odbierze 
                      choć część pliku.
                \item Podobnie w przypadku zakłóceń zewnętrznych (szumów, krzyków,
                      itp.) nie ma gwarancji, że nawet w~przypadku powodzenia 
                      nadawania urządzenie odbierające poprawnie odbierze plik.
            \end{enumerate}
        \item Po zakończeniu nadawania urządzenie nadające poinformuje 
              o~zakończeniu nadawania.
        \item W~aplikacji można wybrać opcję odbierania pliku przez telefon, 
              który staje się wtedy urządzeniem odbierającym.
        \item Po wybraniu opcji odbierania urządzenie odbierające rozpoczyna 
              nasłuchiwanie, czy odbywa się jakieś nadawanie.
        \item W~przypadku nie znalezienia urządzenia nadającego w~ciągu 
              ustawionego czasu (domyślnie 10 sekund) użytkownik zostaje o~tym 
              fakcie poinformowany, a~odbieranie zostaje przerwane.
        \item W~przypadku znalezienia choć jednego urządzenia odbierającego, 
              rozpoczyna się odbieranie.
            \begin{enumerate}
                \item W~sytuacjach opisanych w~punktach E.3. i~E.4., a~także 
                      w~sytuacji, gdy jest więcej niż jedno urządzenie nadające 
                      w~zasięgu, nie ma gwarancji, czy urządzenie odbierze plik 
                      poprawnie.
                \item Urządzenie odbierające może w~przypadku takich problemów 
                      poinformować o~błędzie odbierania i~zakończyć działanie 
                      w~trybie odbierania.
                \item Urządzenie odbierające może również odebrać plik niepoprawny.
            \end{enumerate}
        \item Po zakończeniu odbierania urządzenie prosi o~podanie nazwy pliku, 
              pod którą zapisze odebrany plik w~katalogu utworzonym przez aplikację. 
              W~przypadku niepoprawnej (niezgodnej z~przyjętym standardem nazywania 
              plików w~systemach Android) nazwy użytkownik zostanie poproszony 
              o~inną nazwę.
    \end{enumerate}

\section{Założenia wydajnościowe}
    \begin{enumerate}
        \item Urządzenie przesyła pliki z~szybkością ok. 64 bitów na sekundę.
        \item Warunki otoczenia nie wpływają na szybkość przesyłania, a~jedynie 
              na skuteczność przesyłania.
    \end{enumerate}
        
\section{Założenia pojemnościowe}
    \begin{enumerate}
        \item Urządzenie nadające może przesyłać na raz jeden plik.
        \item Urządzenie odbierające może odbierać na raz jeden plik.
        \item W przypadku równoczesnego nadawania przez dwa urządzenia nadające 
              może nastąpić błąd odbierania, który skutkuje nieprawidłową reakcją
              urządzenia odbierającego.
        \item Równoczesne odbieranie przez dowolną ilość urządzeń odbierających 
              nie skutkuje błędem.
    \end{enumerate}
        
\section{Wymagania sprzętowe}
    \begin{enumerate}
        \item Aplikacja przeznaczona jest na telefony z systemem Android w~wersji
              5.0 lub wyższej.
        \item Telefon musi ponadto posiadać sprawnie działający głośnik i~mikrofon
              pozwalające odpowiednio nadawać i~odbierać dźwięki o~częstotliwości
              od 220 Hz do 8 kHz
    \end{enumerate}
        
\section{Wymagania środowiskowe}
    \begin{enumerate}
        \item Przesyłanie danych musi odbywać się w~cichym, niezaszumionym miejscu, 
              w~którym znajduje się ośrodek umożliwiający rozchodzenie się fal 
              dźwiękowych (np. nie w~próżni).
        \item Odległość w~jakiej mogą znaleźć się telefony, aby móc poprawnie 
              przesłać plik, zależy od sprawności działania ich głośnika i~mikrofonu
              oraz miejsca w~którym odbywa się przesyłanie danych (rozumianego 
              jako położenie w~przestrzeni i~własności fizyczne ośrodka, w~którym
              następuje wymiana danych), przy czym jest ona charakterystyczna 
              dla każdej pary telefonów, a~nie pojedynczego telefonu.
    \end{enumerate}

\end{document}